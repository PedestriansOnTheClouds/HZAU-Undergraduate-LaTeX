\section{材料与方法}这节用来讨论研究XX过程中用到的的材料与方法(\textbf{在本模板中,该节以动量定理的推导与动量守恒定律为例,展示定理,引理等常用论文环境。})
 
\subsection{编号环境与不编号环境}
这一小节我们展示带编号的动量定理推导过程。
\subsubsection{定理带编号的环境}

我们在研究XX问题时,往往需要用到动量定理,在推导动量定理时,我们首先要用到牛顿第二运动定律作为引理:
\begin{lemma}[牛顿第二运动定律]\label{lem2_1} 

    物体加速度的大小跟作用力成正比,跟物体的质量成反比,且与物体质量的倒数成正比;加速度的方向跟作用力的方向相同.使用公式表示即为:
    $$ \vec{F}\propto m\vec{a} $$
    通常我们将比例系数设为1,即为:
    $$\vec{F}= m\vec{a}$$
    
\end{lemma}

由牛顿第二定律,我们可以推导出动量定理如下:
\begin{theorem}[动量定理]\label{thm2_2}
    假设$\vec{p}$为物体的动量,$\vec{F}$为作用在物体上的力的矢量和,则有:
    $$\vec{F}=\frac{d\vec{p}}{dt}$$
\end{theorem}

\begin{proof}
	由牛顿第二定律可得:$$ \vec{a}=\frac{\vec{F}}{m}$$
	由$\vec{a}=\frac{d\vec{v}}{dt}$可得:
	$$\frac{d\vec{v}}{dt}=\frac{\vec{F}}{m}$$
	移项,可得:
	$$\vec{F}=\frac{md\vec{v}}{dt}$$
	即为:
	$$\vec{F}=\frac{d\vec{p}}{dt}$$
\end{proof}

由此我们提出以下命题:
\begin{problem}[动量是否守恒]
	一个系统不受外力或所受外力之和为零的情况下,这个系统的总动量是否保持不变?
\end{problem}

\begin{solution}
    保持不变,可设计实验验证如下:······
\end{solution}

\begin{example}[完全弹性碰撞]
	例如,在完全弹性碰撞的情形下,两小球同时满足动量守恒与动能守恒:
	 $$\mathrm{m}_{1} v_{1}+\mathrm{m}_{2} v_{2}=\mathrm{m}_{1} v_{1}^{\prime}+\mathrm{m}_{2} v_{2}^{\prime}; $$
	$$\frac{1}{2} m_{1} v_{1}^{2}+\frac{1}{2} m_{2} v_{2}^{2}=\frac{1}{2} m_{1} v_{1}^{\prime 2}+\frac{1}{2} m_{2} v_{2}^{\prime 2} ;$$
\end{example}

\subsubsection{定理无编号的环境} %若需要引理,定理等无编号,在begin后的环境中添加“ * ”即可

我们在研究XX问题时,往往需要用到动量定理,在推导动量定理时,我们首先要用到牛顿第二运动定律作为引理:
\begin{lemma*}[牛顿第二运动定律] 
	
	物体加速度的大小跟作用力成正比,跟物体的质量成反比,且与物体质量的倒数成正比;加速度的方向跟作用力的方向相同.使用公式表示即为:
	$$ \vec{F}\propto m\vec{a} $$
	通常我们将比例系数设为1,即为:
	$$\vec{F}= m\vec{a}$$
	
\end{lemma*}

由牛顿第二定律,我们可以推导出动量定理如下:
\begin{theorem*}[动量定理]
	假设$\vec{p}$为物体的动量,$\vec{F}$为作用在物体上的力的矢量和,则有:
	$$\vec{F}=\frac{d\vec{p}}{dt}$$
\end{theorem*}

\begin{proof}
	由牛顿第二定律可得:$$ \vec{a}=\frac{\vec{F}}{m}$$
	由$\vec{a}=\frac{d\vec{v}}{dt}$可得:
	$$\frac{d\vec{v}}{dt}=\frac{\vec{F}}{m}$$
	移项,可得:
	$$\vec{F}=\frac{md\vec{v}}{dt}$$
	即为:
	$$\vec{F}=\frac{d\vec{p}}{dt}$$
\end{proof}

由此我们提出以下命题:
\begin{problem*}[动量是否守恒]
	一个系统不受外力或所受外力之和为零的情况下,这个系统的总动量是否保持不变?
\end{problem*}

\begin{solution}
	保持不变,可设计实验验证如下:······
\end{solution}

\begin{example*}[完全弹性碰撞]
	例如,在完全弹性碰撞的情形下,两小球同时满足动量守恒与动能守恒:
	$$\mathrm{m}_{1} v_{1}+\mathrm{m}_{2} v_{2}=\mathrm{m}_{1} v_{1}^{\prime}+\mathrm{m}_{2} v_{2}^{\prime}; $$
	$$\frac{1}{2} m_{1} v_{1}^{2}+\frac{1}{2} m_{2} v_{2}^{2}=\frac{1}{2} m_{1} v_{1}^{\prime 2}+\frac{1}{2} m_{2} v_{2}^{\prime 2} ;$$
\end{example*}