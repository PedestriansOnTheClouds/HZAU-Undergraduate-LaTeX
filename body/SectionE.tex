\section{注释与引用}\textbf{本模板中,这节用来展示注释与引用。}

\subsection{注释——脚注与尾注}
\subsubsection{脚注}
\par 这里是脚注测试\footnote{1111111111}这里是脚注测试这里是脚注测试这里是脚注测试\footnote{2222222222}这里是脚注测试这里是脚注测试这里是脚注测试这里是脚注测试这里是脚注测试这里是脚注测试这里是脚注测试这里是脚注测试这里是脚注测试这里是脚注测试这里是脚注测试这里是脚注测试这里是脚注测试这里是脚注测试这里是脚注测试\footnote{3333333333}这里是脚注测试这里是脚注测试这里是脚注测试这里是脚注测试这里是脚注测试这里是脚注测试这里是脚注测试这里是脚注测试这里是脚注测试这里是脚注测试这里是脚注测试这里是脚注测试

\par \textcolor{red}{\textbf{注意!正如这份演示中所出现的情况,若该页(也就是本文档中的前一页)剩余空间不大,不足以显示足够多的文档与脚注,那么该段文字就会被移至下一页而留下空白。目前我们尚未找到解决的方法,所以如果遇到了这个问题,请修改排版,以留下足够大的空间。}}

\subsubsection{尾注}
\par 这里是尾注测试\endnote{尾注测试1}这里是尾注测试这里是尾注测试这里是尾注测试这里是尾注测试\endnote{尾注测试2}这里是尾注测试这里是尾注测试这里是尾注测试这里是尾注测试这里是尾注测试这里是尾注测试这里是尾注测试这里是尾注测试这里是尾注测试这里是尾注测试这里是尾注测试这里是尾注测试这里是尾注测试这里是尾注测试这里是尾注测试\endnote{尾注测试3}这里是尾注测试这里是尾注测试这里是尾注测试这里是尾注测试这里是尾注测试这里是尾注测试这里是尾注测试这里是尾注测试这里是尾注测试

\par \textcolor{red}{\textbf{注意!endnotes宏包并不支持hyperref,也就是无法通过点击文中尾注标号以跳转到尾注。当然,这在打印出来的文档中并不会造成任何影响。}}
\par \textcolor{blue}{\textbf{提示:尾注出现在全文最后。为了区分脚注与尾注的编号,我们在尾注编号前加上了“尾注”二字。}}

\subsection{交叉引用}
\par 本模板使用cleveref宏包来进行交叉引用。使用的指令为$\backslash$cref$\{$label$\}$。例子如下:
\par 由\cref{lem2_1}我们可以知道XXXXXXXX。
\par 由\cref{thm2_2}我们可以知道XXXXXXXX。
\par 请注意,label是需要手工设置的,一般将label放在你需要引用的环境内即可(具体可见SectionB.tex),当你点击编译好的pdf文档中被引用的定理时,可以直接跳转到定义原定理的位置。

\subsection{文献引用的演示}
\par 与原作者不同的是,本模板使用bibtex进行文献管理,这是一套较传统的系统(原作者使用的是biblatex进行文献管理)。引用方法如下:
\par 在这里介绍您的研究,并引用文献,比如引用文献 \cite{knuth1984texbook} ,引用文献 \cite{lamport1986latex},引用文献\cite{111},引用文献\cite{222},引用文献\cite{333},引用文献\cite{444},引用文献\cite{clark2017techreport},引用文献\cite{miller2021website},引用文献\cite{green1999misc},引用文献\cite{lee2008phdthesis},引用文献\cite{white2015manual},引用文献\cite{adams2006inbook}

%smith1998book
