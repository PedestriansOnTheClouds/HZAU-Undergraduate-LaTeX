% 该文件是“前言”部分的模板,section, subsection, subsubsection, paragraph, subparagraph为逐级递减的标题,你可以根据自己的需要调整

\section{前言}前言应包括:研究背景与意义、文献综述(国内外研究进展)、研究目的等基本内容。(\textbf{本模板中,这节用来展示文章的5层结构。事实上,一般来说文章层次在3-4层为宜。在之后的section中,我们会只使用至多3层结构(即,节-小节-子节)来进行各种演示。在实际使用中请按照自身需求更改})
 
\subsection{研究背景与意义}这一小节我们介绍XX的研究背景与意义,提出论文研究所针对的科学、生产和经济建设的问题,指出研究这些问题的意义。
\subsubsection{研究背景} 这一子节介绍XX的研究背景······
\subsubsection{研究意义} 这一子节介绍XX的研究意义······

\subsection{XX的国内外研究进展}这一小节我们介绍XX的国内外研究进展(相当于文献综述),主要回顾与所研究课题相关的学科背景,相关领域的研究进展和存
在的问题等。

\subsection{研究目的} 这一小节我们介绍XX的研究目的,研究目的是在提出问题和综述文献的基础上,阐述学术思想,提出科学假设和假说,提出论文研究要实现的目标或达到的目的。
\subsubsection{研究目的的子节} 这一子节中我们将介绍关于研究目的的这些内容······
\paragraph{子节的段标题}这一段我们介绍这些内容(用于演示段标题)······
\subparagraph{小段标题}这一小段我们介绍这些内容(用于演示小段标题)······

